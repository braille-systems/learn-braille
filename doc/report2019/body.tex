\documentclass[main.tex]{subfiles}
\begin{document}
\section{Вступление. Постановка задачи}
\section{Описание и обоснование алгоритмов}
\subfile{parts/nnet.tex}
\newpage
\section{Заключение}
В работе рассмотрены две линейные классификационные модели (однослойный перцептрон и логистическая регрессия) и одна нелинейная (нейронная сеть). Как показано в разделе «описание алгоритмов», на основе тренировочных данных $\{(x_i,y_i)\}$ можно построить модель, наилучшим образом приближающую этот набор данных. Под «наилучшим образом» подразумевается набор параметров модели $\theta$, дающий минимум функции стоимости $J(\theta|\{(x_i,y_i)\})$. Функция стоимости характеризует отклонение предсказаний классификатора от настоящих значений классов на тренировочном наборе данных.

Для линейных моделей удаётся задать выпуклую функцию стоимости (среднеквадратичное отклонение для перцептрона и кросс-энтропия в случае логистической регрессии), которая при заданном наборе $\{(x_i,y_i)\}$ имеет только один минимум. Как следствие, градиентный спуск при любых начальных значениях параметров за конечное число шагов может найти точку, в которой вектор параметров отличается от оптимального не более чем на любую наперёд заданную малую величину. Впрочем, необходимое число итераций заранее неизвестно, поэтому на практике итерационный процесс прерывают, когда норма вектора градиентов $\norm{\nabla J\left(\theta\right)}$ становится меньше некоторого порогового значения, которое выбирается сообразно с требуемой точностью и вычислительными ресурсами.

Функция стоимости нейронной сети в общем случае невыпуклая и содержит множество локальных минимумов. Тем не менее, часто удаётся найти достаточно глубокий локальный минимум. Нейронная сеть может формировать границы со сложным контуром, в то время как линейные модели применимы только к задачам с линейно разделимыми классами. В отдельных случаях задача с не линейно разделимыми классами может быть сведена к таковой путём трансформации пространства входных признаков, но чаще всего вид этой трансформации заранее неизвестен и применить её нельзя. Поэтому нейронные сети представляют собой гораздо более мощный и гибкий инструмент, чем линейные модели. К тому же способность образовывать границу сложной формы позволяет серьёзно увеличивать точность классификатора при добавлении новых точек в тренировочный набор данных – в отличие от линейных моделей, где граница лишь будет сдвинута, нейросеть может дать совсем иную конфигурацию параметров.
На следующем этапе работы планируется применить алгоритмы к распознаванию изображения как совокупности пикселей. Также планируется рассчитать несколько численных метрик изображений и попытаться обучить классификаторы, распознающие символы азбуки Брайля по набору метрик картинки.

\end{document}